\documentclass[aspectratio=43,unicode,10pt]{beamer}
\usetheme{ttipresentation}

\usepackage{luatexja}
\usepackage{luatexja-fontspec}
\usepackage{graphicx}
\usepackage{multicol}

\setmainjfont{ipagp.otf}
\beamertemplatenavigationsymbolsempty

\newcommand{\itemtitle}[1]{\textbf{#1}\\}
\newcommand{\fire}[1]{\textcolor{red}{\textbf{#1}}}
%\newcommand{\freeze}[1]{\textcolor{blue}{\textbf{#1}}}
\newcommand{\then}{\textcolor{ttiblue}{\textbf{⇒}}\hspace{1ex}}
\newcommand{\good}{\textcolor{orange}{\textbf{◎}}\hspace{1ex}}
\newcommand{\arrow}{\textcolor{ttiblue}{\textbf{→}}\hspace{1ex}}
\newcommand{\mb}[1]{\mathbf{#1}}
\newcommand{\arrowed}[1]{\vec{\mathbf{#1}}}
\newcommand{\opennt}{未完成の非終端記号}


\newcommand{\thetitle}{Recurrent Neural Network Grammers}
\title{\thetitle}
\institute{知能数理研究室}
\author{外山洋太、三輪誠、佐々木裕}
\date{\today}


\newcommand{\bigger}{\Large}

\begin{document}

\begin{frame}
  \titlepage
\end{frame}

\begin{frame}{導入}
  \begin{block}{\thetitle~(RNNGs)}
    \begin{itemize}
      \item 文の確率的生成モデル
        \begin{itemize}
          \item \fire{単語や句の入れ子的・階層的構造を陽に表現}
        \end{itemize}
      \item 問題:構文解析、文生成
      \item 動機:SequentialなRecurrent Neural Networks (RNNs)は \\
            自然言語の潜在的な入れ子構造を考慮できていない
      \item 構文解析(文 \rightarrow~構文木)と文生成のアルゴリズム
    \end{itemize}
  \end{block}
\end{frame}

\begin{frame}{提案手法}
  \begin{block}{RNNGの定義}
    \begin{gather*}
      RNNG := (N, \Sigma, \Theta) \\
      N \cup \Sigma = \emptyset \\
      \begin{cases}
        N: \text{非終端記号の有限集合} \\
        \Sigma: \text{終端記号の有限集合} \\
        \Theta: \text{NNのパラメータ}
      \end{cases}
    \end{gather*}
  \end{block}
\end{frame}

\begin{frame}{提案手法}
  \begin{block}{構文解析のアルゴリズム}
    \begin{gather*}
      f: X \rightarrow Y \\
      \begin{cases}
        x: \text{単語列(入力)} \\
        y: \text{構文木(出力)} \\
        S: \text{スタック} \\
        B: \text{入力バッファ}
      \end{cases}
    \end{gather*}
    \begin{itemize}
      \item 入力バッファの要素:まだ使われていない終端記号
      \item スタックの要素:終端記号、\opennt、完成した非終端記号
    \end{itemize}
  \end{block}
\end{frame}

\begin{frame}{提案手法}
  \begin{block}{構文解析のアルゴリズム}
    \begin{itemize}
      \item 遷移の制約
        \begin{itemize}
          \item $n$: スタック内の\opennt の数
        \end{itemize}
        \begin{table}
          \begin{tabular}{c | l}
            遷移 & 制約 \\
            \hline
            NT(X)   & $B \neq \emptyset \wedge n < 100$ \\
            \hline
            SHIFT   & $B \neq \emptyset \wedge n \geq 1$ \\
            \hline
            REDUCE  & \parbox{20em}{
              $スタック内の一番上の要素が \\ \opennt でない$ \\
              $\wedge (n \geq 2 \vee B = \emptyset)$
            } \\
          \end{tabular}
        \end{table}
    \end{itemize}
  \end{block}
\end{frame}

\begin{frame}{提案手法}
  \begin{block}{生成のアルゴリズム}
    \begin{itemize}
      \item 遷移の制約
        \begin{itemize}
          \item $n$: スタック内の\opennt の数
        \end{itemize}
        \begin{table}
          \begin{tabular}{c | l}
            遷移 & 制約 \\
            \hline
            GEN(X) & $n \geq 1$ \\
            \hline
            REDUCE  & \parbox{20em}{$
              スタック内の一番上の要素が \\
              \opennt でない \\
              \wedge n \geq 1
            $} \\
          \end{tabular}
        \end{table}
    \end{itemize}
  \end{block}
\end{frame}

% \begin{frame}{\thetitle}
%   \only<1>{
%     \begin{itemize}
%       \bigger
%       \item タスク:商品レビューの\fire{レーティング予測}
%       \item \fire{レビュー解析}:レビューのどの部分が重要か
%     \end{itemize}

%     \begin{figure}
%       \includegraphics[width=0.8\linewidth]{fig/review.pdf}
%     \end{figure}
%   }

%   \only<2>{
%     \begin{itemize}
%       \bigger
%       \item 対象言語:以下の種類の文字を含む言語
%     \end{itemize}
%     \begin{figure}
%       \includegraphics[width=0.75\linewidth]{fig/logogram_and_ideogram.pdf}
%     \end{figure}
%     \begin{itemize}
%       \bigger
%       \item モデル: \fire{フォント画像} \arrow 文字埋め込み \\
%                     \arrow 単語、文、文書埋め込み \arrow レーティング
%     \end{itemize}
%   }

%   \only<3>{
%     \begin{itemize}
%       \bigger
%       \item 解析結果
%     \end{itemize}
%     \begin{figure}
%       \includegraphics[width=0.8\linewidth]{fig/review_1.png}
%     \end{figure}\vspace{-2ex} % HACK
%     \begin{figure}
%       \includegraphics[width=0.7\linewidth]{fig/review_2.png}
%     \end{figure}
%   }
% \end{frame}

\end{document}
